\section{Wstęp}
	W latach 90 dwudziestego wieku gry komputerowe zaczęły budzić zainteresowanie coraz większej ilości osób między innymi z powodu wprowadzania tytułów posiadających grafikę w trójwymiarze (3D). Był to kamień milowy, który udowodnił, że istnieje możliwość odtworzenia rzeczywistości w świecie wirtualnym. Niestety, ograniczeniem była wtedy wydajność sprzętowa.\par
	Dziś, dwie dekady później, technologia jest na bardzo wysokim poziomie. Powstały gry takie jak \textit{Grand Theft Auto V}, czy \textit{Wiedźmin 3: Dziki Gon}, które zachwycają oprawą graficzną, rozgrywką, muzyką oraz pozwalają graczowi poczuć, że niekiedy tamta, wirtualna rzeczywistość, bywa bardziej realistyczna od tej prawdziwej.\par
	Przeskok technologiczny i rozwój Internetu pozwoliły nie tylko na granie w pojedynkę (singleplayer), ale również na rozgrywkę z innymi (multiplayer). To, w połączeniu z ludzką naturą, chęcią rywalizacji, utworzyło nową kategorię sportu o nazwie \blockquote{e-sport}.\par
	E-sport jest formą rywalizacji, która odbywa się za pośrednictwem gier komputerowych. Jej głównym celem jest wygrana którą można osiągnąć tylko dzięki taktyce i koordynacji drużyny. Aby uzyskać odpowiedź dlaczego e-sport jest tak popularny należy zadać pytanie: Dlaczego piłka nożna jest najpopularniejszą dyscypliną sportu na świecie? \cite{footballPopularity} W odróżnieniu, np. od hokeja, który wymaga lodu, łyżew, krążka, kija i ochraniaczy, aby zagrać w piłkę nożną wystarczy bardzo niewiele: piłka i dwa słupki. Istnieje teoria zwana \textit{Prawem Bushnell-a}, która brzmi następująco:
	\begin{center}
		\blockquote{All the best games are easy to learn and difficult to master.} \cite{burshnellsLaw}
	\end{center}
Co w wolnym tłumaczeniu znaczy, że najlepsze gry to takie, które są proste w nauce, jednak trudne w opanowaniu. Taka właśnie jest piłka nożna i taki jest e-sport, gdzie aby naśladować profesjonalistę wystarczą jedynie chęci.\par
	Profesjonalni sportowcy muszą liczyć się z naturalnym zmęczeniem organizmu, dlatego trenują tylko kilka godzin dziennie. Natomiast profesjonalni \blockquote{e-sportowcy}, aby być w światowej czołówce trenują często po 12 godzin dziennie lub więcej. \cite{proPlayerPlayTime} Czas ten nieznacznie różni się w przypadku zwykłych graczy komputerowych. Oczywiście, część gra 2 godziny, a część spędza cały dzień. Nie zmienia to jednak faktu, że w obu tych przypadkach po pewnym czasie zaczyna się odczuwać monotonię. Sama wygrana przestaje być celem nadrzędnym. Przez to cierpi nie tylko sam gracz, ale i jego zespół, który nie będąc w pełni zgranym staje się bardziej podatny na porażkę.\par
	\subsection*{Cel pracy}
		Aby rozwiązać problem znudzenia i braku efektywności w rozgrywce celem pracy jest zmotywowanie gracza do wygranej oraz polepszenie jego umiejętności poprzez wykonywanie zadań. Zadania nie zagwarantują wygranej, lecz w mniejszym lub większym stopniu do niej przybliżą. Użytkownik będzie miał możliwość tworzenia i personalizowania zadań do wykonania w wybranych grach. Określić będzie mógł ile razy ma je wykonać oraz do kiedy. Aby móc sprawdzić czy zadanie zostało wykonane w pierwszej kolejności musi rozegrać prawdziwy mecz w wybranej grze.\par
	
	
	