\subsection{Wielojęzyczność}
	Gry e-sportowe nie ograniczają się do wąskiej grupy ludzi z jednego kraju. Zrzeszają graczy z całego świata, którzy często znają tylko jeden język. Dlatego gry te posiadają wsparcie dla wielu języków, ponieważ użytkownik nie będzie szukał gry, do której będzie się musiał dostosować, a raczej takiej, która się dostosuje do niego. W tej aplikacji również nie ograniczono się do jednego języka, a rozszerzalność na kolejne odbywa się na zasadzie przetłumaczenia pliku z tłumaczeniami i dodanie w opcjach nowego języka.

	\subsubsection*{Biblioteka \blockquote{angular-translate}}
		Do wprowadzenia wielojęzyczności użyta została biblioteka \blockquote{angular-translate} \cite{ref:angularTranslateDoc}. Jest to moduł do frameworka AngularJS, który ma przydatne w użyciu dyrektywy, filtry i serwisy, oraz posiada fazę konfiguracji, w której można spersonalizować działanie za pomocą dostępnych opcji.\par
		 Podczas startu aplikacji, aby wybrać język, w pierwszej kolejności sprawdzany jest Local Storage. Następnie sprawdzany jest obecny język przeglądarki. Jeśli obie te metody zawiodą, to wybrany zostaje język domyślny (angielski). Nowo wybrany język jest automatycznie zapisywany do Local Storage.\par
		 Każdy język posiada tłumaczenie w oddzielnym pliku w formacie JSON. Rozdziela to funkcjonalność od reszty kodu w aplikacji. Inną zaletą jest ułatwienie zarządzania treścią, ponieważ łatwiej jest manipulować sformułowaniami w jednym pliku, niż w bardzo wielu nieczytelnych widokach HTML. Przykład takiego tłumaczenia można zobaczyć w kodzie źródłowym \ref{lis:translation}. W przypadku braku konkretnego klucza w tłumaczeniu wyświetlany jest komunikat w  docelowym miejscu na stronie oraz w konsoli (kod źródłowy \ref{lis:missingTranslation}). Informuje on o tym, który klucz należy uzupełnić i w jakim języku. Informacja ta jest lokalna, ale w razie rozwoju aplikacji można by w takim wypadku wysyłać ją do serwera, który poinformuje dewelopera o błędach.
		 
\begin{code}[
		language=javascript,
		caption={Problem synchronizacji stanu aplikacji w jQuery (źródło: opracowanie własne)},
		label={lis:jquery-state-sync-problem},
		escapechar=|
	]
var user = { |\label{line:jquery-user}|
  name: "John",
  age: 25
}

function updateDOMUser() { |\label{line:jquery-update-dom}|
  \$("#userName").text(user.name)
  \$("#userAge").text(user.age)
}

user = getNewUser() |\label{line:jquery-user-get}|
updateDOMUser()

	\end{code}\\
	
	W kodzie źródłowym \ref{lis:jquery-state-sync-problem} przedstawiony został ... W linijce \ref{line:jquery-user} widać obiekt usera.
	
\clearpage

\begin{table} % you can add [!ht] here to make this table always float to the top
		\centering
		\caption{Parametry sprzętu, na którym przeprowadzone zostały testy aplikacji webowych (źródło: opracowanie własne)}
		\label{tab:comp-spec}
    	\begin{tabular}{ | l | l |}
    \hline
    Nazwa & Wersja \\ \hline
    Procesor & Intel Core i7-8700k 3,7GHz \\ \hline
    Pamięć RAM & Corsair Venegeance LPX DDR4 3200MHz 32GB \\ \hline
    Dysk twardy & Crucial 250 GB 2,5'' SATA SSD \\ \hline
    System Operacyjny & Windows 10 Educational 64 bit, wersja 1803 \\ \hline
    Przeglądarka & Google Chrome 64 bit, wersja 66.0.3359.181 \\ \hline
    \end{tabular}
\end{table}

	W tabelce \ref{tab:comp-spec} pokazane zostały parametry sprzętu.